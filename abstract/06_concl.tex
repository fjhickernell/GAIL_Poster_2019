\section{Conclusions and Future Work}
\label{sec:concl}
We have presented a short overview of GAIL with some motivating
examples. GAIL and the research behind the software are still under active
development. We list some of the potentially fruitful
directions to pursue in order to strengthen the software.

For the existing algorithms, it would be worthy to conduct or develop
large-scale tests such as work of
~\cite{dolan2002benchmarking,floudas2013handbook} for optimization
problems, to challenge the software for performance in terms of success
rates, execution speed, and memory usage. These test suites could serve to
be a more rigorous basis of comparisons with more established solvers  or
to validate future incremental changes in our own.  

Instead of using \texttt{fminbnd} in Example~\ref{eg1}, it would be better
to use solvers that are designed for global optimization such as those from
 MATLAB's Global Optimization Toolbox (GOT). That said, GOT does not
guarantee returning global minimum; see~\cite{GOT}. Likewise, for 
Example~\ref{eg4}, more extensive study can be done. Another
idea is to price Asian options with GAIL cubatures and also
\texttt{asianbyls} from MATLAB's Financial Instruments Toolbox.

We are upgrading Bayesian cubature algorithm to use Sobol' points and Walsh
kernel whereas the current version uses rank-1 lattice points and
shift-invariant kernel. One major advantage is that Walsh kernel based
algorithm does not require the integrand to be periodic, which opens up
more possibilities. Additional improvements being developed are to add
gradient in parameter searching; this will allow using gradient descent
method instead of heuristic search, and could possibly speed up the algorithm.
Shift-invariant kernels used in the algorithm are currently limited to order
of even integers which is usually fixed. Enhancements are being developed
to avoid this constraint.

Each GAIL algorithm could find itself into many high-impact application
areas such as computational finance, machine learning, decision sciences.
Reliable cubatures are particularly important for valuing financial
derivatives. GAIL already contains a few option-pricing functions,  
namely, \texttt{assetPath}, \texttt{optPayoff}, and  \texttt{optPrice}. 
However, they have yet to elevated as a first-class citizens in
the software library because they need to be accompanied by detailed documentation
and unit tests suites.

Currently, most of the GAIL algorithms are written and released in Matlab. Other computer languages have gained popularity in scientific computing. Given the flexibility of the GAIL software design, GAIL could very well be ported to support Python, Julia, and R.
\begin{comment}
\begin{enumerate}

\item \texttt{assetPath}: A class of discretized stochastic processes that
model the values of an asset with respect to time

\item \texttt{optPayoff}: A class of option payoffs based on asset paths

\item \texttt{optPrice}: A class that computes the price of an option via
(quasi-)Monte Carlo methods.

\end{enumerate}


\alnote{Is there any future to support another popular scientific computing
language? like Julia ? Should we mention Fred's idea to bring other
collaborators? Or mention here interested collaborators contact us at--- ?}
\scnote{QMC Community software and OO classes}
\end{comment}
