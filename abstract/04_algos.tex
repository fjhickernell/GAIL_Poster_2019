\section{Algorithms}
\label{sec:algo}

GAIL software developers periodically release a well-tested package
including recently developed algorithms by the collaborators. Most notable
algorithms in the GAIL's latest release, version 2.3, shall be grouped into
two major categories.
\begin{enumerate} [label=\Roman*.]

\item Approximation and optimization:

    \begin{enumerate}[label={[\arabic*]}]

    \item \texttt{funappx\_g}: One-dimensional function approximation on
    bounded interval~\cite{FunappxFunmin}

    \item \texttt{funmin\_g}: Global minimum value of univariate function
    on a closed interval~\cite{FunappxFunmin}

    \end{enumerate}

\item Numerical Integration:

    \begin{enumerate}[label={[\arabic*]}, start=3, resume*]

    \item \texttt{integral\_g}: One-dimensional integration on bounded
    interval~\cite{Z18}

    \item \texttt{meanMC\_g}: Monte Carlo (MC) method for estimating mean
    of a random variable~\cite{meanMCcubMC}

    \item \texttt{meanMC\_CLT}: MC method with Central Limit Theorem (CLT)
    confidence intervals for estimating mean of a random variable
    \cite{ChoEtal18b}

    \item \texttt{cubMC\_g}: MC method for  multiple integration
    \cite{meanMCcubMC}

    \item \texttt{cubSobol\_g}: Quasi-Monte Carlo (QMC) method using Sobol'
    cubature for multiple integration \cite{cubSobol}

    \item \texttt{cubLattice\_g}: QMC method using rank-1 lattices cubature
    for multiple integration \cite{cubLattice}

    \item \texttt{cubBayesLattice\_g}: Bayesian cubature method using
    lattice sampling for multiple integration \cite{cubBayesLattice}

    \end{enumerate}

\end{enumerate}

The GAIL algorithms, denoted \texttt{<algo>} below, have the following
three interface patterns:  
\begin{enumerate}  [label=\roman*.]
\item \texttt{[approx, out\_param] = <algo>(f, in\_param)} 
\item \texttt{<algo>(f, array, abstol, ...)} 
\item \texttt{<algo>(f, \textquotesingle param\textquotesingle, value, ...)}.
\end{enumerate}
Outputs, \texttt{approx} and \texttt{out\_param}, store the approximate
solution and a structure of input and other outputs from the solvers. The
algorithms require only function input \texttt{f}. Other important inputs
are provided with default values. 
%\alnote{Does the following statement generalize for all GAIL algorithms?}
For example, tolerance $\varepsilon$ is
set to $10^{-6}$ for all univariate GAIL algorithms in the absence of user input. 
The first pattern involves
an optional structure of input values. The second pattern takes a list of
optional inputs such as \texttt{array} that defines domain $\mathcal{D}$
and \texttt{abstol} that stores user tolerance. The last pattern allows
users to invoke the solver with key-value pairs of parameter names and
values.


We follow the philosophy of reproducible research advocated by \cite{SEP1, 
BD95} and software citation principles by~\cite{smith2016software}.
In addition, we borrow sustainable practices of scientific software
development from~\cite{KCNN15,KCWH16,KCLM14}. We are convinced that true
scholarship in computational sciences are realized by reliable
reproducibility~\cite{C14a,CH13}. And so we use the best software
engineering practices available in MATLAB, for example,  unit tests, doc
tests, and searchable HTML documentation. For collaboration and
productivity, we adopt agile software development using tools such as
Trello and version control systems with GitHub. 

